\documentclass[letterpaper]{article}
\usepackage[letterpaper, left=0.625in, right=0.625in, bottom=1in, top=0.75in]{geometry}
\usepackage{cmap}
\usepackage{multicol}
\usepackage[T1]{fontenc}
\usepackage{lmodern}
\usepackage{amsmath}

\title{Rapport de stage}
\author{Nathan Boyer}

\begin{document}

\maketitle


\begin{multicols}{2}
    


\begin{abstract}
    
    It's possible to distribute the Internet to users via drones.
    However, this raises the question of how to place the drones around the users, and how to distribute the bandwidth between the different users.
    A reinforcement AI has already been designed to address this problem.
    However, in this article, we will see how learning and optimization can be combined to further improve performance.

\end{abstract}


\section{Probleme presentation}

We have m users, classified into 3 categories, each class having its own bandwidth demand for a drone.
We then want to place n drones, and for each drone, decide how much of its bandwidth it gives to each class of drone, so that as many drones as possible are satisfied, i.e.\:the bandwidth available to them is greater than or equal to their demand.

\section{Problem solving by constrained optimization}

\subsection{Rigourous problem definition}

This problem can be solved by constrained optimization.

First, we have to define the equation that computes how well a user will reveive a base station's connection depending on where it is.

We have the following Equations for the drones's SINRs, which is a factor that represents how much data is effectively receiveed by the user, in relation to the data sent by the base station to the user:

$\operatorname{SINR}_{i, j}^t=\frac{p c \mu\left(y_j, u_i^t\right)\left(\left(\left\|y_j-u_i^t\right\|\right)^2+\left(h\right)^2\right)^{-\alpha / 2}}{\sum\limits_{k \in \mathcal{U} \backslash i} p c \mu\left(y_j, u_k^t\right)\left(\left(\left\|y_j-u_k^t\right\|\right)^2+\left(h\right)^2\right)^{-\alpha / 2}+\sigma^2}$

To evaluate a configuration we will thus proceed as follows:

You first need to associate each user with its nearest base station.

You then share the bandwidth of a base station dedicated to a given class between all users of this class associated to this base station.

Then, you compute each user SINR and decide for the user is satisfied or not.

The percentage of satisfied user is thus what you want to maximize.

\subsection{Optimization problem}

This problem can be resolved by as an optimization problem, I will show you how to resolve it for 2 base stations.

Let's first define u a matrix of dimensions $N*N$, with N being the number of positions possible for a base station.
$u_{i,j}$ is a binary variable that is equal to one if and only if the first base station is in position i and the second is position j.

$bw$ is a $3*2$ matrix that represents the bandwidth each base station allocates to each user class.

$\delta$ is a vector that represents for every whether it is satisfied or not.

The function to optimize is then: $\max_{u,bw}\sum_{g\in\mathcal{G}}\delta_g$

Under the following constraints:

\begin{enumerate}
    \item $\sum_{b \in |\mathcal{U}|} \sum_{i, j \in \mathcal{U}} u_{i, j} \times bw_{\text{slice}(g), b} \times \text{bps}_g \frac{100}{G_{\text{conn}}(\text{slice}(g),b)} \nonumber$
    $\geq th_g \times \delta_g \hspace{1cm} \forall g \in \mathcal{G} \\[1em]$
    \item $bw_{em,b} + bw_{ur,b} + bw_{mm,b} = 1 \hspace{1cm} \forall b \in |\mathcal{U}| \\[1em]$
    \item $bw_{em,b} \geq 0; bw_{ur,b} \geq 0; bw_{mm,b} \geq 0 \hspace{1cm} \forall b \in |\mathcal{U}| \\[1em]$
    \item $\sum_{i,j \in \mathcal{U}}u_{i,j} = 1 \\[1em]$
    \item $\delta_g \in \{0, 1\} \hspace{1cm} \forall g \in \mathcal{G} \\[1em] $
    \item $u_{i,j} \in \{0, 1\} \hspace{1cm} \forall i,j \in \{0, 1, \ldots , |\mathcal{U}-1|\}$
\end{enumerate}

The resolution of this optimization problem can be implemented in Python with the help of a module such as pulp.

\subsection{Time Optimizations}

To speed up the resolution of this problem, I implemented a few optimizations:

First, only positions in the convex enveloppe of the users can actually be candidates for optimal placements of the base stations.
We can thus reduce by a lot the size of the matrix u.

Moreover, it is possible to go further but not without losing precision.
Indeed, you can divide the users into k clusters and take the union of this convex enveloppe of this clusters.
The graph of performance and time in function of the number of clusters can be found below.

However, since the computation times are pretty reasonable for 2 base stations, I decided to go with only 1 cluster.

Finally, multithreading can be used to further accelerate the computations.

\section{Apprentissage pour résoudre le problème}

\subsection{Apprentissage par réenforcement}

\subsection{Apprentissage par rapport à la solution optimum}

\subsection{Mixer les 2}

\subsection{Les graph neural network}

\section{Ouverture: Utiliser la distance à l'optimum comme erreur pour l'apprentissage}

\section{Conclusion}

\end{multicols}

\end{document}


